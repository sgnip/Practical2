\documentclass[11pt]{report}

\usepackage{fancyhdr} % Cabeceras de página
\usepackage{lastpage} % Módulo para añadir una referencia a la última página
\usepackage{titling} % No tengo claro para qué es esto
\usepackage[left=3cm,right=2.5cm,top=3cm,bottom=2cm]{geometry} % Márgenes
\usepackage[T1]{fontenc}
\usepackage[utf8x]{inputenc}
\usepackage{xspace}
\usepackage{graphicx}
\usepackage{tikz}
\usepackage{wrapfig}
\usepackage{hyperref}
\usepackage{amssymb}
\usepackage{multirow}
\usepackage[official]{eurosym}
\usepackage{enumitem}
\usepackage{pdfpages}

\hypersetup{
  	hyperindex,
    colorlinks,
    allcolors=blue!60!black
}


\setcounter{secnumdepth}{3}
\renewcommand{\baselinestretch}{1.4}

\title{UAM Software Notification and Damage Management System \\ Project Plan}
\date{\today}
\author{{\Large Triforce} \\ \vspace{5pt} \textit{Iván Márquez Pardo, Víctor de Juan Sanz, Guillermo Julián Moreno}}

\fancyhf{}
\fancypagestyle{plain}{%
	\lhead{\raisebox{12pt}{\textsc{Project Plan} - \small Ref. TFC-UAM-01}}
	\chead{\centering \vspace{-15pt} \includegraphics[width =40 pt]{../Logo.jpg}}
	\rhead{\raisebox{12pt}{\small Ver. 1.0 - \today \vspace{2pt}}}
	\cfoot{\thepage\ of \pageref{LastPage}}
	\rfoot{}
}

\newcounter{reqs}[chapter]
\newcounter{NFreqs}[chapter]
\newcommand{\header}[1]{\\ \indent \textbf{#1}\hspace{10pt}}

\newcommand{\reqvertsep}{\vspace{-7pt}}

\newcommand{\reqdesc}{\subparagraph{Description}}
\newcommand{\reqin}{\reqvertsep\subparagraph{Input data}}
\newcommand{\reqout}{\reqvertsep\subparagraph{Output data}}
\newcommand{\reqsteps}{\reqvertsep\subparagraph{Steps}}

\newenvironment{requirement}[1]{
	\refstepcounter{reqs}
	\par {\noindent \bfseries {\textsc{Functional Requirement \arabic{reqs}} - #1}} \par
}{\vspace{20pt}}

\newenvironment{NFrequirement}[1]{
	\par {\noindent \bfseries {\textsc{Non-Functional Requirement \arabic{NFreqs}}}} \par
	\refstepcounter{NFreqs}
}{\vspace{20pt}}

\newcommand{\reqref}[1]{Req. \ref{#1}}

\begin{document}
\maketitle

\begin{table}[hbtp]
\centering
\begin{tabular}{|c|c|p{3cm}|p{3.5cm}|}
\hline \multicolumn{4}{|c|}{\textsc{Version table}} \\ \hline \hline
\textbf{Version} & \textbf{Date} & \textbf{Main changes} & \textbf{Purpose} \\ \hline
\end{tabular}
\end{table}

\begin{description}
\item[Written by] Guillermo Julián Moreno, Víctor de Juan Sanz, Iván Márquez Pardo.
\item[Reviewed by] Guillermo Julián Moreno, Víctor de Juan Sanz, Iván Márquez Pardo (\today).
\item[Approved by] Guillermo Julián Moreno, Víctor de Juan Sanz, Iván Márquez Pardo (\today).
\end{description}

\newpage

\begin{abstract}
In order to improve their performance, more and more maintenance services in charge of different entities or organizations have started to automate their labor using technologies and software systems specially designed taking into account the needs this sector has.

In this document, we outline and detail the project plan for the Fault Manager Lite system, a proposal for the UAM that will solve their problems regarding maintenance efficiency. An estimate for the size, time and cost of the system will be derived and presented, together with a plan and schedule to make this project real.
\end{abstract}

\tableofcontents
\newpage
\listoftables
\newpage

\pagestyle{plain}

\chapter{Introduction}
\label{chapIntroduction}

The Autonomous University of Madrid (UAM) has reported the numerous problems it has detecting faults that arise on its campus and its facilities, whose reparation usually takes excessive time and is poorly organized. A late detection of faults in the facilities delays its reparation, stopping its users to continue using them as normal and complicating the maintenance staff labor. Regarding the wishes of the campus users and realizing the potential maintenance problems it has, the UAM has organized a contest to choose the best project proposal that solves them.

This is where our organization, Triforce, enters the scene: we have analyzed the problem exhaustively and designed a web application, Fault Manager Lite (FML), that meets all the requirements expected, solves the problems and also includes new extra features that makes it even more useful.

\section{Purpose}

The purpose of this Project Plan document is to provide a detailed description of the requirements for the project and estimations of the size of the whole system. Based on these aspects, we introduce our proposed planning for the development, including related estimates such as cost and duration.

The intended audience of this document is the executive team of Triforce, in order to know and evaluate the resource allocation for the project and its fit in the company and its activities.

\section{Structure}

The document is structured as follows. This chapter serves as introduction and description of the document and its contents. Chapter \ref{chapSubsystems} contains the definition of the subsystems that will conform the project and their associated requirements.

In chapter \ref{chapProjectSizeEstimations} we present the estimations for the size of the system. Our planned schedule and cost estimations is presented in chapter \ref{chapPlan} and, finally, chapter \ref{chapTracking} shows the proposal for the project tracking and control.

The appendices contain extra information, such as detailed function point estimation (appendix \ref{chapFunctionPoints}), Cocomo II Cost estimation (appendix \ref{chapCocomo}) and the detailed Gantt diagram (appendix \ref{chapGantt}).

\chapter{Subsystems and Requirements}
\label{chapSubsystems}
\input{tex/Subsystems.tex}
% -*- root: ../ProjectPlan.tex -*-

\chapter{Requirements}
\label{chapRequirements}

The functional and non-functional requirements of the Fault Manager Lite application are gathered in the next sections of the document.

Functional requirements are organized into their corresponding subsystems. If any requirement could depend on two or more subsystems due to the function it performs, then one of those subsystems (usually the most relevant) will be chosen.

In a similar way, non-functional requirements are organized into categories attending to their scope.

\section{Functional Requirements}

\subsection{Task Management}

\begin{requirement}{Generate repair task}
\reqdesc When the report system receives a new fault report, it will analyze the information it contains in order to generate its corresponding repair task. This task will later be automatically assigned to a technician.
\reqin The fault report sent by an user of the application.
\reqsteps The system will analyze the information of this fault report and generate its corresponding repair task. That task will be classified depending on its category and priority, and saved in the fault database.
\reqout The generated repair task, whose status will be 'Pending'.
\end{requirement}

\begin{requirement}{Visualize new notices of incidences}
\reqdesc Managers will be able to see new notices of incidences that have not been assigned yet, in order to revise their correctness.
\reqin Function triggered manually by the manager by selecting its corresponding option.
\reqsteps Depending on the department the manager is in charge of, the system will look in the fault database for new notices of incidences of that department that are still not solved nor assigned to any technician.
\reqout The system will display a list of incidence reports. Information about each report will also be shown to see their correctness.
\end{requirement}

\begin{requirement}{Definition of priority criteria}
\reqdesc The system must allow to specify different criteria to assign automatically a priority to each incidence.
\reqin The user selects parameters of the incidence object and corresponding keywords that will conform the selection criteria, and the desired priority that will be set on match.
\reqsteps The system stores the criteria and applies it automatically on incidence creation in order to establish a priority.
\reqout The system automatically classifies each incidence based on user-defined criteria: if a certain parameter (e.g., location or description) contains the keywords specified, the priority will be set to whatever the user specified in the criteria.
\end{requirement}

\begin{requirement}{Management of priorities by managers}
\reqdesc Managers must be able to receive reports and manage the incidences based on the departments they are in charge of.
\reqin The maintenance personnel sends messages and incidences to the system.
\reqsteps The system saves all the information and shows them to the personnel in charge.
\reqout Managers sees all related messages and reports.
\end{requirement}

\begin{requirement}{Changes of incidence state}
\reqdesc Managers must be able to change the state (Pending, Assigned or Solved) of each incidence of their departments.
\reqin THe manager introduces the new state of the incidence.
\reqsteps The system saves the new state in the database.
\reqout The incidence has the new state registered.
\end{requirement}

\begin{requirement}{Changes of incidence category}
\reqdesc Managers must be able to change the category of each incidence.
\reqin The personnel in charge inputs the new category of the incidence.
\reqsteps The system saves the new category.
\reqout The incidence has the new category registered.
\end{requirement}

\begin{requirement}{Reassignment of tasks}
\reqdesc Managers must be able to change the technician assigned to a repair task.
\reqin The manager has to select a repair task from his/her department and introduce the identifier of other member of his/her department
\reqsteps The system will look into the staff database for the identifier of the new technician and update the corresponding field in the selected repair task.
\reqout Changes in the database entry of the repair task will be saved. The new technician will receive a notification of the new task he/she is in charge of; the previous technician will also receive a task canceling notification.
\end{requirement}

\begin{requirement}{Modify notices of incidences}
\reqdesc Managers will be able to modify reports of incidences for their departments in order to correct any mistake or complete gaps in the forms.
\reqin The manager will be able to change any of the fields (except for the identifier) of a fault report by introducing new information or modifying the existing one. Among the modifiable fields of the report, there will be: title, description, location (in case automatic allocation has failed), category and priority. Confirmation will be needed to save changes.
\reqsteps The system will update in the faults database the data given/modified by the manager. All the modifications of the report will be done within a transaction, and only be applied if the manager confirms so when he/she finishes the process.
\reqout Changes on the fault database will be committed becoming permanent (successful transaction) and the manager will be notified with a confirmation message on the screen.
\end{requirement}

\begin{requirement}{Manual assignment of incidences}
\reqdesc Managers of each department will be able to assign the reparation of incidences to technical members of their department.
\reqin The manager will have to introduce the identifier of the fault report and the name/identifier of the technician that will fix that fault.
\reqsteps The system will look for the technician in the staff database and for the fault report in the faults database. The status of the fault will be changed from 'Pending' to 'Assigned' and the fault will now be related to its technician.
\reqout A confirmation message will be shown to the manager and the technician will receive a notification of the new task he/she has been assigned to repair.
\end{requirement}

\begin{requirement}{Automatic assignment of incidences}
\reqdesc The system will automatically assign repair tasks whose status is 'Pending' to the closest technician from the department of the fault (category).
\reqin A repair task with 'Pending' status.
\reqsteps The system will take the category and location of the reported fault and look for the closest technician that belongs to that department and assign him/her that repair task. The status of the fault will be changed from 'Pending' to 'Assigned' and the fault will now be related to its technician.
\reqout The status of the fault in the database will be changed and the technician will receive a notification of the new task he/she has been assigned to repair.
\end{requirement}

\begin{requirement}{Filtering list of categories}
\reqdesc The system must allow users to search and filter from the category list.
\reqin Logical combination (\textit{and, or, not}) of attribute and value pairs that constitute the filter.
\reqsteps The system validates the filter query, then filters the list of categories according to that filter.
\reqout List of filtered categories.
\end{requirement}

\begin{requirement}{Add new categories}
\reqdesc The system must allow administrators to add new categories.
\reqin Information needed to create category. This information is given by external inputs: title and information of department's chief.
\reqsteps The system ensures that the category does not exist previously before creating it, and then saves the category in the database.
\reqout The department chief object is stored into database and the list of categories is updated.
\end{requirement}

\begin{requirement}{Delete categories}
\reqdesc The system must allow administrators to delete categories.
\reqin The category object to delete.
\reqsteps The system finds the category and every task with the category being deleted. Those tasks are changed removing this category to maintain coherence of the inner system database. Finally, the category is removed from the system.
\reqout The category disappears from the database and from every task categorized with it.
\end{requirement}

\begin{requirement}{Modifying category stored values and its priority}
\reqdesc The system must allow administrators to modify categories attributes and its priority stored.
\reqin The category to be modified and the attributes that are going to be changed.
\reqsteps The system finds the category and changes the corresponding attributes. There's no need to update related objects as the category identification number is not modifiable, except if the priority is modified.

If the priority is modified, the system will update the priority of every task that hasn't been set manually by the users.
\reqout The database is updated with the new information for the category. If the priority for the category is modified, tasks in the system are updated accordingly and the system will notify the users related with those updated tasks.
\end{requirement}


\subsection{Report System}

\begin{requirement}{Fault location}
\label{reqFaultLocation}
\reqdesc Get the location of the fault, based on the location of the reporter.
\reqin Two kind of external inputs are necessary for determine location:
\begin{itemize}
\item Facility where the fault is located, given by.
\subitem GPS coordinates (if possible).
\subitem Manually chose location.
\item Manually chose floor and room inside the building where the fault is located.
\end{itemize}
\reqsteps Transform input information into Location object.
\reqout Location object.
\end{requirement}

\begin{requirement}{Incidences report}
\reqdesc The system must allow users to report incidences.
\reqin The user must input a title and a category for the report. The system will automatically set a location for the report (\reqref{reqFaultLocation}). Optionally, the user can add a description of the problem and a picture. The system will send also the identification of the user who's reporting the task.
\reqsteps The system saves in the database a new record with the input data and the user who reported it. The system adds additional calculated information such as priority (decided automatically based on category).
\reqout The reported fault object is stored into the database, and the maintenance personnel is notified of the new fault. The user receives an acknowledgment of the received report.
\end{requirement}

\subsection{Notification and Messaging System}

\begin{requirement}{Notification when the fault has been solved}
\reqdesc The user will receive a notification when a fault he has reported has been solved.
\reqin Reported fault object and the user who will receive the notification.
\reqsteps The system retrieves the notification preferences for the user (e.g., mail, push mobile notification), builds the notification and sends it through the corresponding external channel.
\reqout A message saying the fault has been solved successfully with a brief summary of it is sent to the user.
\end{requirement}

\begin{requirement}{Messaging between users}
\reqdesc The personnel in charge must be able to talk with the reporter of the incidence.
\reqin The report to talk about and the message to send.
\reqsteps The system retrieves the users that should receive the message. Those are the users related to that task (reporters and maintenance personnel) except the one who sent the message. The message is stored in the system and every user will see it in the chat section of the selected report.
\reqout All users related to the task can see the new message sent regarding the corresponding report.
\end{requirement}


\subsection{User Management}

\begin{requirement}{See fault history}
\reqdesc Registered users will be able to list the faults they have reported and select one of them.
\reqin The user that wants to see his/her tasks.
\reqsteps The system retrieves the list of faults reported by the given user.
\reqout A list of the reports sent by the given user and ordered by date.
\end{requirement}

\begin{requirement}{Show to a manager the list of members of his/her technical staff}
\reqdesc Managers will be able to visualize an interactive list of members of the technical staff they are in charge of. The list will be interactive in the sense that the manager would be able to see more information about an specific technician (see next requirement).
\reqin The manager has to introduce the name and/or the numeric identifier of the technician whose statistics and profile wants to see. Then, the manager will have to select one from the results of the search done.
\reqsteps The system will use the criteria introduced by the manager to look for employees that match them. Then, it will show a list of the matchings and once the manager has chosen one of the list, the system will gather the corresponding information from the staff database (profile info) and the fault database (in order to generate statistics about the chosen technician).
\reqout The system will show the manager the profile of the selected technician. In addition, it will also generate some statistics related to the work performed by that technician.
\end{requirement}

\begin{requirement}{Show detailed information about a technician}
\reqdesc Managers will have available to get more information about any of the technicians that are members of their departments.This information will be in the form of statistics and profiles of the corresponding technicians.
\reqin The manager has to choose one of the technicians from the complete list of members of his/her technical staff (see previous) whose statistics and profile wants to see.
\reqsteps The system will retrieve the corresponding information about this technician from the staff database (personal and contact information) and the fault database (tasks assigned and solved tasks in order to generate statistics about his/her performance). The statistics generated will include information about (at least) his/her average working time, effectiveness and most common tasks solved.
\reqout The system will show the manager the profile of the selected technician. In addition, it will also generate some statistics related to the work performed by that technician. This information will be given to the manager in the form of a report.
\end{requirement}



\subsection{Fault History and Statistics}


\begin{requirement}{Check history of reported fault}
\reqdesc Registered users will be able to check details of a reported fault.
\reqin Fault reported object with all information attached.
\reqout Report showing all the information attached to the object.
\end{requirement}
\begin{requirement}{Tracking incidence impact}
\reqdesc The system must allow internal users with a maintenance personnel profile to check the impact of the incidences.
\reqin Incidence object associated to the user logged in.
\reqsteps The system retrieves the incidence from the database, specifically the impact saved.
\reqout Numeric value showing the impact of the incidence.
\end{requirement}

\begin{requirement}{Get report of incidences}
\reqdesc Managers will have access to a report on incidences. On one hand this report will show the incidences that are not solved or those that are being solved in order of priority. On the other hand, the incidences report will show in a similar way the last incidences that have been solved, sorted by
their resolution date.
\reqin The manager will trigger this action by selecting its corresponding option.
\reqsteps The system will look into the faults database for fault reports that are not solved or they are being solved at the moment, and will order the results by priority. Then, in the same database, the system will filter by reports already solved and order them by resolution date. With these two sets of results, an interactive report will be produced.
\reqout The system will show the manager the report produced by his/her request. This report is interactive in the sense that the manager can select one of the faults to see its characteristics.
\end{requirement}

\begin{requirement}{List all reports}\label{A4-ListAllReports}
\reqdesc The system must allow listing all reported faults.
\reqin
\begin{itemize}
 	\item Some (or none) values to filter the list by, such as:
 		\subitem Category.
 		\subitem Priority.
 		\subitem Location.
 \end{itemize}
\reqsteps The system will search into the database all reports matching the given values. If a value is changed, the list must be filtered again with the new given values.
\reqout External \textbf{query}: List of the reports matching with given values, ordered mainly by category and secondly by date. This list can be updated interactively by the user modifying filter values.
\end{requirement}

\begin{requirement}{Consult single report}
\reqdesc Users will be able to check details of a reported fault from the full list of reports (given by \ref{A4-ListAllReports}).
\reqin Fault reported object with all information attached.
\reqout Report showing all the information attached to the object.
\end{requirement}


\begin{requirement}{Visualize history of tasks}
\reqdesc Managers will be able to visualize the complete
history of tasks in a table where each row represents an incidence.
\reqin The manager will trigger this function by selecting its corresponding option.
\reqsteps The system will access the incidences database in order to retrieve information about all the faults reported and show it to the manager. By default, the tasks will be ordered by date.
\reqout The system will display in the screen a table containing all the incidences reported until that moment.
\end{requirement}

\begin{requirement}{Sort history of tasks}
\reqdesc Managers will be able to sort the history of tasks depending of different criteria chosen by the manager. This sorting criteria includes at least: priority, category, date, state.
\reqin The complete history of tasks has been previously retrieved. The manager will have to choose a criteria among the list given above.
\reqsteps The system will take the results of the history of tasks it has previously retrieved and order them by the chosen criteria.
\reqout The system will display in the screen a table containing all the incidences sorted in the requested order.
\end{requirement}

\begin{requirement}{Consult extended information about an incidence on the report}
\reqdesc Managers will be able to consult extended information on each of the incidences that appear in the report of incidences.
\reqin Report of incidences produced by the system. The manager will have to select one of the incidences of the report to see detailed information about it.
\reqsteps The system will look into the faults database for the selected fault, gather all the information related to it and show it to the manager.
\reqout The system will show the manager a new screen with detailed information about the incidence he/she has selected in the report.
\end{requirement}

\begin{requirement}{Generate statistics about incidences}
\reqdesc Managers will be able to get statistics related to the tasks and to their related areas will be available in order to
detect problematic areas or departments.
\reqin The manager will trigger this function by selecting its corresponding option.
\reqsteps The system will take all the information about incidences stored in the incidences database in order to analyze it and generate some statistics about them. Relevant information for the statistics will be the category, priority and location of tasks. Statistics will also take into account the costs of the repairs and the time elapsed from the detection of a fault until it has been solved (statistics about effectiveness and efficiency of fault troubleshooting).
\reqout A report containing the statistics generated will be shown to the manager.
\end{requirement}

\begin{requirement}{Print reports/screens}
\reqdesc All the reports generated by the application will be printable. This also applies to new screens that the application opens in order to show any requested information.
\reqin Report or screen that the system has generated or opened by the request of an user.
\reqsteps The system will convert the report or the screen to a PDF document and send it to the printer.
\reqout The printed document.
\end{requirement}


\section{Non-Functional Requirements}



\chapter{Project Size Estimations}
\label{chapProjectSizeEstimations}
% -*- root: ../ProjectPlan.tex -*-
\section{Size estimation}

Estimations of time and complexity of the system are basic in order to make a project plan adjusted to the real development time. On the contrary, we won't have any basis to aproximate that time, depending just on a correct planification guided by previous experiences.

With this aim, the size of the Fault Manager Lite project has been estimated using the Function Points estimation technique. Moreover, we have also estimated the Report System (subsystem of FML) with the COCOMO II tool.

\section{Estimation using function points}
\subsection{Unadjusted Function Points}
After the evaluation of the requirements of the proposed system using the Function Points analysis, we have obtained the data of its estimated complexity. We attach tables with the size of the different subsystems in which the application is divided into, with their corresponding estimations in terms of Function Points. For a more detailed analysis of each of the exposed requirements, see \ref{chapFunctionPoints}.


\subsubsection{Task Management Subsystem}
Table \ref{tbl_TMS_UFP} shows the summary of the unadjusted Function Points calculated for the Task Management Subsystem.
\begin{table}[hbtp]
\centering
\begin{tabular}{|l|c|c|c|c|c|c|c|}
\cline{2-7}
\multicolumn{1}{c}{} & \multicolumn{6}{|c|}{\textsc{Complexity}} & \multicolumn{1}{c}{}  \\ \cline{2-8}
\multicolumn{1}{c|}{} & \textbf{Low} & \textbf{Medium} & \textbf{High} & \textbf{Low} & \textbf{Medium} & \textbf{High} & \multirow{2}{*}{\textit{Unadjusted FP}} \\ \cline{1-7}
\textbf{Data fns.} & \multicolumn{3}{|c|}{\textit{Frequency}} &  \multicolumn{3}{|c|}{\textit{Weight}} & \\ \hline
ILF 	& 16 & 0 & 0 & 7 & 10 & 15 & 112 	\\ \hline
EIF 	& 1  & 0 & 0 & 5 & 7  & 10 & 5		\\ \hline
\textbf{Transactional fns.} & \multicolumn{7}{|c|}{} \\ \hline
EI 		& 2  & 0 & 0 & 3 & 4  & 6  & 6 		\\ \hline
EO 		& 0  & 3 & 0 & 4 & 5  & 7  & 15		\\ \hline
EQ		& 15 & 0 & 0 & 3 & 4  & 6  & 45		\\ \hline
\multicolumn{6}{c|}{} & \textbf{Total} & 183 \\ \cline{7-8}
\end{tabular}
\caption{Overview of the calculation of the Unadjusted Function Points for the Task Management Subsystem.}
\label{tbl_TMS_UFP}
\end{table}

\subsubsection{Report System Subsystem}
Table \ref{tbl_RSS_UFP} shows the summary of the unadjusted Function Points calculated for the Report System Subsystem.
\begin{table}[hbtp]
\centering
\begin{tabular}{|l|c|c|c|c|c|c|c|}
\cline{2-7}
\multicolumn{1}{c}{} & \multicolumn{6}{|c|}{\textsc{Complexity}} & \multicolumn{1}{c}{}  \\ \cline{2-8}
\multicolumn{1}{c|}{} & \textbf{Low} & \textbf{Medium} & \textbf{High} & \textbf{Low} & \textbf{Medium} & \textbf{High} & \multirow{2}{*}{\textit{Unadjusted FP}} \\ \cline{1-7}
\textbf{Data fns.} & \multicolumn{3}{|c|}{\textit{Frequency}} &  \multicolumn{3}{|c|}{\textit{Weight}} & \\ \hline
ILF 	& 16 & 0 & 0 & 7 & 10 & 15 & 112 	\\ \hline
EIF 	& 1  & 0 & 0 & 5 & 7  & 10 & 5		\\ \hline
\textbf{Transactional fns.} & \multicolumn{7}{|c|}{} \\ \hline
EI 		& 2  & 0 & 0 & 3 & 4  & 6  & 6 		\\ \hline
EO 		& 0  & 3 & 0 & 4 & 5  & 7  & 15		\\ \hline
EQ		& 15 & 0 & 0 & 3 & 4  & 6  & 45		\\ \hline
\multicolumn{6}{c|}{} & \textbf{Total} & 183 \\ \cline{7-8}
\end{tabular}
\caption{Overview of the calculation of the Unadjusted Function Points for the Report System Subsystem.}
\label{tbl_RSS_UFP}
\end{table}

\subsubsection{Notification and Messaging System Subsystem}
Table \ref{tbl_NMSS_UFP} shows the summary of the unadjusted Function Points calculated for the Notification and Messaging System Subsystem.
\begin{table}[hbtp]
\centering
\begin{tabular}{|l|c|c|c|c|c|c|c|}
\cline{2-7}
\multicolumn{1}{c}{} & \multicolumn{6}{|c|}{\textsc{Complexity}} & \multicolumn{1}{c}{}  \\ \cline{2-8}
\multicolumn{1}{c|}{} & \textbf{Low} & \textbf{Medium} & \textbf{High} & \textbf{Low} & \textbf{Medium} & \textbf{High} & \multirow{2}{*}{\textit{Unadjusted FP}} \\ \cline{1-7}
\textbf{Data fns.} & \multicolumn{3}{|c|}{\textit{Frequency}} &  \multicolumn{3}{|c|}{\textit{Weight}} & \\ \hline
ILF 	& 16 & 0 & 0 & 7 & 10 & 15 & 112 	\\ \hline
EIF 	& 1  & 0 & 0 & 5 & 7  & 10 & 5		\\ \hline
\textbf{Transactional fns.} & \multicolumn{7}{|c|}{} \\ \hline
EI 		& 2  & 0 & 0 & 3 & 4  & 6  & 6 		\\ \hline
EO 		& 0  & 3 & 0 & 4 & 5  & 7  & 15		\\ \hline
EQ		& 15 & 0 & 0 & 3 & 4  & 6  & 45		\\ \hline
\multicolumn{6}{c|}{} & \textbf{Total} & 183 \\ \cline{7-8}
\end{tabular}
\caption{Overview of the calculation of the Unadjusted Function Points for the Notification and Messaging System Subsystem.}
\label{tbl_NMSS_UFP}
\end{table}

\subsubsection{User Management Subsystem}
Table \ref{tbl_UMS_UFP} shows the summary of the unadjusted Function Points calculated for the User Management Subsystem.
\begin{table}[hbtp]
\centering
\begin{tabular}{|l|c|c|c|c|c|c|c|}
\cline{2-7}
\multicolumn{1}{c}{} & \multicolumn{6}{|c|}{\textsc{Complexity}} & \multicolumn{1}{c}{}  \\ \cline{2-8}
\multicolumn{1}{c|}{} & \textbf{Low} & \textbf{Medium} & \textbf{High} & \textbf{Low} & \textbf{Medium} & \textbf{High} & \multirow{2}{*}{\textit{Unadjusted FP}} \\ \cline{1-7}
\textbf{Data fns.} & \multicolumn{3}{|c|}{\textit{Frequency}} &  \multicolumn{3}{|c|}{\textit{Weight}} & \\ \hline
ILF 	& 16 & 0 & 0 & 7 & 10 & 15 & 112 	\\ \hline
EIF 	& 1  & 0 & 0 & 5 & 7  & 10 & 5		\\ \hline
\textbf{Transactional fns.} & \multicolumn{7}{|c|}{} \\ \hline
EI 		& 2  & 0 & 0 & 3 & 4  & 6  & 6 		\\ \hline
EO 		& 0  & 3 & 0 & 4 & 5  & 7  & 15		\\ \hline
EQ		& 15 & 0 & 0 & 3 & 4  & 6  & 45		\\ \hline
\multicolumn{6}{c|}{} & \textbf{Total} & 183 \\ \cline{7-8}
\end{tabular}
\caption{Overview of the calculation of the Unadjusted Function Points for the User Management Subsystem.}
\label{tbl_UMS_UFP}
\end{table}

\subsubsection{Faults History and Statistics Subsystem}
Table \ref{tbl_FHSS_UFP} shows the summary of the unadjusted Function Points calculated for the Faults History and Statistics Subsystem.
\begin{table}[hbtp]
\centering
\begin{tabular}{|l|c|c|c|c|c|c|c|}
\cline{2-7}
\multicolumn{1}{c}{} & \multicolumn{6}{|c|}{\textsc{Complexity}} & \multicolumn{1}{c}{}  \\ \cline{2-8}
\multicolumn{1}{c|}{} & \textbf{Low} & \textbf{Medium} & \textbf{High} & \textbf{Low} & \textbf{Medium} & \textbf{High} & \multirow{2}{*}{\textit{Unadjusted FP}} \\ \cline{1-7}
\textbf{Data fns.} & \multicolumn{3}{|c|}{\textit{Frequency}} &  \multicolumn{3}{|c|}{\textit{Weight}} & \\ \hline
ILF 	& 16 & 0 & 0 & 7 & 10 & 15 & 112 	\\ \hline
EIF 	& 1  & 0 & 0 & 5 & 7  & 10 & 5		\\ \hline
\textbf{Transactional fns.} & \multicolumn{7}{|c|}{} \\ \hline
EI 		& 2  & 0 & 0 & 3 & 4  & 6  & 6 		\\ \hline
EO 		& 0  & 3 & 0 & 4 & 5  & 7  & 15		\\ \hline
EQ		& 15 & 0 & 0 & 3 & 4  & 6  & 45		\\ \hline
\multicolumn{6}{c|}{} & \textbf{Total} & 183 \\ \cline{7-8}
\end{tabular}
\caption{Overview of the calculation of the Unadjusted Function Points for the Faults History and Statistics Subsystem.}
\label{tbl_FHSS_UFP}
\end{table}

\subsubsection{Global Estimation}
Table \ref{tbl_GLOBAL_UFP} shows the summary of the unadjusted Function Points of the whole system, given by the sum of the unadjusted FP of each subsystem.
\begin{table}[hbtp]
\centering

\begin{tabular}{|l|c|c|c|c|c|c|c|}
\cline{2-7}
\multicolumn{1}{c}{} & \multicolumn{6}{|c|}{\textsc{Complexity}} & \multicolumn{1}{c}{}  \\ \cline{2-8}
\multicolumn{1}{c|}{} & \textbf{Low} & \textbf{Medium} & \textbf{High} & \textbf{Low} & \textbf{Medium} & \textbf{High} & \multirow{2}{*}{\textit{Unadjusted FP}} \\ \cline{1-7}
\textbf{Data fns.} & \multicolumn{3}{|c|}{\textit{Frequency}} &  \multicolumn{3}{|c|}{\textit{Weight}} & \\ \hline
ILF 	& 15 & 1 & 0 & 7 & 10 & 15 & 113 	\\ \hline
EIF 	& 1  & 0 & 0 & 5 & 7  & 10 & 5		\\ \hline
\textbf{Transactional fns.} & \multicolumn{7}{|c|}{} \\ \hline
EI 		& 2  & 0 & 0 & 3 & 4  & 6  & 6 		\\ \hline
EO 		& 0  & 3 & 0 & 4 & 5  & 7  & 15		\\ \hline
EQ		& 16 & 0 & 0 & 3 & 4  & 6  & 48		\\ \hline
\multicolumn{6}{c|}{} & \textbf{Total} & 189 \\ \cline{7-8}
\end{tabular}

\caption{Overview of the calculation of the Unadjusted Function Points for the whole system.}
\label{tbl_GLOBAL_UFP}
\end{table}

\emph{The number of Unadjusted Function Points of the system is XXXX.}


\subsection{Adjusting Factor}
Once the unadjusted Function Points of the project have been calculated, these must be ponderated depending of the technical characteristics of the system. To do this, we have evaluated the 14 adjusting characterístics involved in the Function Points method in order to extract the Adjusting Factor for this system.

Table \ref{tbl_ADJ_FACTOR} shows the summary of the values assigned to the general characteristics of the system.

\begin{table}[hbtp]
\centering
\begin{tabular}{l|c}
\textbf{Complexity Factors} & \textbf{Value} \\ \hline
Data communication & 4 \\
Distributed functions & 3 \\
Performance & 1 \\
Strongly used configurations & 2 \\
Transactions frequency & 4 \\
Online data input & 4 \\
Design for the efficiency of the final user & 3 \\
Online update & 3 \\
Complex processes & 3 \\
Other systems utilization & 3 \\
Installation ease & 5 \\
Operation ease & 2 \\
Multiple sites instalation & 3 \\
Change ease & 5 \\ \hline
\textit{TOTAL DEGREE OF INFLUENCE (TDI)} & \textit{45.0}
\end{tabular}
\caption{Complexity Factor values.}
\label{tbl_FPEstimation}
\end{table}

Therefore, the TDI is 45.0. The Adjusting Factor (AF) is calculated applying the corresponding formula:
\emph{AF = (TDI x 0.01) + 0.65 = 1.10}

\emph{The Adjusting Factor of the system is 1.10.}

\subsection{Adjusted Function Points}
Once we have obtained the Unadjusted Function Points, UFP, and the Adjusting Factor, AF, we have that the Adjusted Function Points, AFP, are:
\emph{AFP = UFP x AF}

This way, we have that the AFP of each subsystem, as well as the total AFP for the project are the ones exposed in Table \ref{tbl_GLOBAL_AFP}.

\begin{table}[hbtp]
\centering
\begin{tabular}{l|c|c}
\textbf{Subsystem} & \textbf{UFP} & \textbf{AFP} \\ \hline
Task Management  & X & X \\
Report System & X & X \\
Notification and Messaging System & X & X \\
User Management & X & X \\
Faults History and Statistics & X & X \\
Raw function points (FP) & X & X \\ \hline
\textit{Total} & \textit{325.1} & \textit{325.1}
\end{tabular}
\caption{Adjusted Function Points organized by subsystems.}
\label{tbl_GLOBAL_AFP}
\end{table}



\begin{table}[hbtp]
\centering
\begin{tabular}{l|c}
\textbf{Summary} & \textbf{Value} \\ \hline
Raw function points (FP) & 186 \\
Complexity factor (CF) & 45 \\
Adjustment factor & 1.10 \\
Adjusted function points & 204.6 \\
Person/day per function points & 1.615 \\ \hline
\textit{Estimated person-days} & \textit{330.429}
\end{tabular}

\caption{Estimation of the project size based on the function points detailed in table \ref{tbl_GLOBAL_UFP}.}
\label{tbl_FPEstimation}
\end{table}

Tables \ref{tbl_GLOBAL_UFP} and \ref{tbl_FPEstimation} detail the estimation of the project size and time required for its completion based on the method of function points. Based on these calculations, the project would consist of approximately 207 adjusted function points. That would translate to 335 person-days of development time.

\subsection{Estimation using CoCoMo II method}

I don't know how to use Cocomo to estimate size.



\chapter{Project Plan}
\label{chapPlan}
% -*- root: ../ProjectPlan.tex -*-


\section{Gantt diagram}

\section{Resources and activities assignment}

\section{Increment planning}

We have decided to break down the development in \textcolor{red}{TODO: SET THIS} increments, each one dedicated to various subsystems. The details of the function points assigned to each increment and the corresponding effort is detailed in table \ref{tblIncrementsSubsystems}. Table \ref{tblSubsystemsAssignedIncrement} reflects the increment in which each subsystem will be completed and the corresponding percentage of each increment's effort dedicated to it.

\begin{table}[hbtp]
\centering

\begin{tabular}{c|c|c|c}
\textbf{Increment} & \textbf{Subsystems} & \textbf{Function Points} & \textbf{Effort (person-days)} \\ \hline
Increment N & A,B,C & X & Y \\
Increment N & A,B,C & X & Y \\ \hline
\textit{Total} &  & \textit{X} & \textit{Y} \\
\end{tabular}

\caption{Detail of the increments and corresponding effort.}
\label{tblIncrementsSubsystems}
\end{table}

\begin{table}[hbtp]
\centering
\input{../CommonTex/SubsystemAndIncrement.tex}
\caption{Assigned increment and effort for each subsystem.}
\label{tblSubsystemsAssignedIncrement}
\end{table}

The effort for each phase of each increment is detailed in table \ref{tblIncrementPhases}.

\begin{table}[hbtp]
\centering
\input{../CommonTex/IncrementEffort.tex}
\caption{Detail of the increments with the corresponding phases for each one.}
\label{tblIncrementPhases}
\end{table}

\section{Cost estimation}

Given this schedule, cost estimation:

\begin{figure}[hbtp]
\centering
\begin{tabular}{r|rr|r}
\textbf{Resource} & \textbf{Quantity} & \textbf{Unit cost} & \textbf{Total cost} \\ \hline
System Analyst & 848 hours & 400 \euro / day & 42,400 \euro \\
Senior Designer & 504 hours & 350 \euro / day & 22,050 \euro \\
Junior Designer 1 & 504 hours & 200 \euro / day & 12,600 \euro \\
Junior Designer 2 & 504 hours & 200 \euro / day & 12,600 \euro \\
Systems Technician & 128 hours & 300 \euro / day & 4,800 \euro \\ \hline
\textit{Total work} & \textit{1,994 hours} & - & \textit{94,450 \euro} \\ \hline \hline
Systems Maintenance & 128 days & 1,050 \euro / month & 6,109 \euro \\
IDE Software & 3 workstations & 1,100 \euro / w.s. & 3,300 \euro \\
Development w.s. & 3 workstations & 1,650 \euro / w.s. & 4,950 \euro \\
Performance w.s. & 1 workstation & 3,200 \euro / w.s. & 3,200 \euro \\ \hline
\textit{Total systems} & - & - & \textit{17,559 \euro} \\ \hline \hline
\textit{Total cost} & - & - & \textit{112,009 \euro}

\end{tabular}

\caption{Cost estimation given the schedule}
\label{tblCostResourceEstimate}
\end{figure}

% -*- root: ../ProjectPlan.tex -*-
\section{Cost management}

\subsection{Personnel costs}

We consider the following salaries for the required positions for the project:

\begin{itemize}
\item Systems Analyst: 400 \euro / day.
\item Senior Designer: 350 \euro / day.
\item Junior Designer: 200 \euro / day.
\item Systems Technician: 300 \euro / day.
\end{itemize}

These values will be used as the fixed cost for personnel throughout this document estimations.

\subsection{Hardware costs}

For the project, three workstations must be acquired for development (1.650 \euro per unit) and another one for performance testing (3.200 \euro). Total hardware cost is then 8.150 \euro.

Additionally, the maintenance cost for current hardware and software is 1.050 \euro per month.

\subsection{Software costs}

A new integrated development environment will also be acquired, at a cost of 1,100 \euro per workstation. This environment includes all necessary software for the lifecycle of the project.

\section{Estimation of development costs}

\begin{table}[hbtp]
\centering
\begin{tabular}{l|c|cccc}
\multirow{2}{*}{\textbf{Phase}} & \textbf{Effort} & \textbf{Labour} & \textbf{Running} & \textbf{Fixed} & \textbf{Total} \\
& \textit{Person-months} & \textit{K\euro} & \textit{K\euro / month} & \textit{K\euro} & \textit{K\euro} \\
\textsc{Requirements} & 0.7 & 0.8 & 1.05 & 0 & 0.905  \\
\textsc{Product design} & 0.3 & 1.5 & 1.05 & 0 & 1.815  \\
\textsc{Detailed design} & 0.4 & 2.3 & 1.05 & 0 & 2.72  \\
\textsc{Code \& Unit tests} & 0.5 & 2.8 & 1.05 & 8.25 & 11.575  \\
\textsc{Integration \& test} & 0.3 & 1.6 & 1.05 & 3.2 & 5.115  \\ \hline
\textsc{Development} & 1.5 & 8.2 & 1.05 & 11.45 & 21.225  \\
\textsc{Totals} & 1.6 & 9 & 1.05 & 11.45 & 22.13
\end{tabular}

\caption{Detailed report of the estimation using the CoCoMo II method}
\label{tblCocomoDetail}
\end{table}


\chapter{Project Tracking \& Control}
\label{chapTracking}
% -*- root: ../ProjectPlan.tex -*-
\section{Change management}

Details for the changes in the project configuration will be outlined in the Configuration Management Document.

\section{Progress monitorization}

Before the project starts, a meeting is planned for \textcolor{red}{TODO: FIX THIS DATE} where both representatives of Triforce and UAM will review and, if it corresponds, approve the project plan. Then, project will start.

Meetings are planned at each phase (Analysis, Design, Coding, Testing and Deployment) of each increment as reflected on table \ref{tblPlannedMeetings}. Each phase and its deliverables will be reviewed at these meetings.

It is mandatory that the responsible of each phase and the manager are present in each meeting. Other Triforce employees interested can attend with previous approval from the responsible of the meeting.

\textcolor{red}{TODO: FIX MEETING DATES}

\begin{table}[hbtp]
\centering
\begin{tabular}{|c|c|c|c|}
\hline
\textbf{Increment} & \textbf{Activity} & \textbf{Responsible} & \textbf{Date} \\ \hline
\multirow{5}{*}{Increment 1} & Analysis & System Analyst & 0 / 0 / 0 \\ \cline{2-4}
 & Design & Senior Designer & 0 / 0 / 0 \\ \cline{2-4}
 & Coding & Senior Programmer & 0 / 0 / 0 \\ \cline{2-4}
 & Testing & System Analyst & 0 / 0 / 0 \\ \cline{2-4}
 & Deployment & Systems Tech & 0 / 0 / 0 \\ \hline
\end{tabular}
\caption{Planned project meetings.}
\label{tblPlannedMeetings}
\end{table}


\appendix

\chapter{Estimation with Function Points}
\label{chapFunctionPoints}
\setlist{itemsep = 0.3pt}
% -*- root: ../ProjectPlan.tex -*-

In this section we give more detailed information about the elements that have been taken into account in order to determine the function points of each of the subsystems that form the Fault Manager Lite software system. To do it, we will make an exhaustive analysis of the requirements specified in Section \ref{chapRequirements}, indicating for each one, the elements considered for the final calculation.External Query (EQ) 


\section{Task Management}

\subsection{Generate repair task}
\textbf{Internal Logic File (ILF)} \\ 
\textbf{RET/FLR}
\begin{itemize}
\item Fault report in the application database.
\end{itemize}
\textbf{DET}
\begin{itemize}
\item ID of the fault.
\item Location of the fault.
\item Category of the fault.
\item Subject of the fault.
\item Description of the fault.
\item Status of the fault.
\item Priority of the fault.
\item Timestamp of the report.
\item Photography of the fault.
\item ID of the reporter.
\item ID of the assigned technician.
\end{itemize}
\textbf{\underline{Conclusion:}} It has a \textbf{LOW} complexity with 1 RET/FTR and 11 DET.

\subsection{Visualize new notices of incidences}
\textbf{External Query (EQ)} \\ 
\textbf{RET/FLR}
\begin{itemize}
\item List of unassigned reports of incidences.
\end{itemize}
\textbf{DET}
\begin{itemize}
\item ID of each fault.
\item Location of each fault.
\item Category of each fault.
\item Subject of each fault.
\item Description of each fault.
\item Status of each fault.
\item Priority of each fault.
\item Timestamp of each fault report.
\item Photography of each fault.
\item ID of each fault's reporter.
\end{itemize}
\textbf{\underline{Conclusion:}} It has a \textbf{LOW} complexity with 1 RET/FTR and 10 DET.

\subsection{Definition of priority criteria}
\textbf{Internal Logic File (ILF)} \\ 
\textbf{RET/FLR}
\begin{itemize}
\item Application database.
\end{itemize}
\textbf{DET}
\begin{itemize}
\item Up to 15 keywords and parameters that will conform the new priority criteria, specified by the manager.
\end{itemize}
\textbf{\underline{Conclusion:}} It has a \textbf{LOW} complexity with 1 RET/FTR and 15 DET.

\subsection{Management of priorities by managers}
\textbf{External Query (EQ)} \\ 
\textbf{RET/FLR}
\begin{itemize}
\item List of faults in the application database.
\end{itemize}
\textbf{DET}
\begin{itemize}
\item ID of each fault.
\item Location of each fault.
\item Category of each fault.
\item Subject of each fault.
\item Description of each fault.
\item Status of each fault.
\item Priority of each fault.
\item Timestamp of each fault report.
\item Photography of each fault.
\item ID of each fault's reporter.
\item ID of each fault's assigned technician.
\end{itemize}
\textbf{\underline{Conclusion:}} It has a \textbf{LOW} complexity with 1 RET/FTR and 10 DET.

\subsection{Changes of incidence status}
\textbf{Internal Logic File (ILF)} \\ 
\textbf{RET/FLR}
\begin{itemize}
\item Incidence in the application database.
\end{itemize}
\textbf{DET}
\begin{itemize}
\item New status of the incidence.
\end{itemize}
\textbf{\underline{Conclusion:}} It has a \textbf{LOW} complexity with 1 RET/FTR and 1 DET.

\subsection{Changes of incidence category}
\textbf{Internal Logic File (ILF)} \\ 
\textbf{RET/FLR}
\begin{itemize}
\item Incidence in the application database.
\end{itemize}
\textbf{DET}
\begin{itemize}
\item New category of the incidence.
\end{itemize}
\textbf{\underline{Conclusion:}} It has a \textbf{LOW} complexity with 1 RET/FTR and 1 DET.

\subsection{Reassignment of tasks }
We have divide this requirement in two parts. The updating requirement to be more precise and specific with RET and DETs associated.
\subsubsection{Reassignment of tasks (Updating)}
\textbf{Internal Logic File (ILF)} \\ 
\textbf{RET/FLR}
\begin{itemize}
\item Incidence in the application database.
\end{itemize}
\textbf{DET}
\begin{itemize}
\item ID of the new technician assigned to this task.
\end{itemize}
\textbf{\underline{Conclusion:}} It has a \textbf{LOW} complexity with 1 RET/FTR and 1 DET.

\subsubsection{Reassignment of tasks (Notifying)}
\textbf{Internal Logic File (ILF)} \\ 
\textbf{RET/FLR}
\begin{itemize}
\item Application database.
\end{itemize}
\textbf{DET}
\begin{itemize}
\item ID of the new technician assigned to this task.
\item ID of the old technician assigned to this task.
\item Message to the new technician.
\item Message to the old technician.
\item Fault's ID.

\end{itemize}
\textbf{\underline{Conclusion:}} It has a \textbf{LOW} complexity with 1 RET/FTR and 5 DET.

\subsection{Modify notices of incidences}
\textbf{Internal Logic File (ILF)} \\ 
\textbf{RET/FLR}
\begin{itemize}
\item Incidence in the application database.
\end{itemize}
\textbf{DET}
\begin{itemize}
\item Location of the fault.
\item Category of the fault.
\item Subject of the fault.
\item Description of the fault.
\item Status of the fault.
\item Priority of the fault.
\end{itemize}
\textbf{\underline{Conclusion:}} It has a \textbf{LOW} complexity with 1 RET/FTR and 6 DET.

\subsection{Manual assignment of incidence}
\textbf{Internal Logic File (ILF)} \\ 
\textbf{RET/FLR}
\begin{itemize}
\item Incidence in the application database.
\item Application database.
\end{itemize}
\textbf{DET}
\begin{itemize}
\item Fault's ID.
\item ID of the technician assigned.
\end{itemize}
\textbf{\underline{Conclusion:}} It has a \textbf{LOW} complexity with 2 RET/FTR and 2 DET.

\subsection{Automatic assignment of incidences}
\textbf{Internal Logic File (ILF)} \\ 
\textbf{RET/FLR}
\begin{itemize}
\item Incidence in the application database.
\item Application database.
\end{itemize}
\textbf{DET}
\begin{itemize}
\item Fault's ID.
\item ID of the technician assigned.
\end{itemize}
\textbf{\underline{Conclusion:}} It has a \textbf{LOW} complexity with 2 RET/FTR and 2 DET.


\subsection{Filtering list of categories}
\textbf{External Query (EQ)} \\ 
\textbf{RET/FLR}
\begin{itemize}
\item None.
\end{itemize}
\textbf{DET}
\begin{itemize}
\item Heating.
\item Plumbing.
\item Air conditioning.
\item Cleaning (x6 buildings).
\item Information Technologies.
\item Elevators.
\item Electricity.
\item Trash Collection.
\item Gardeners.
\end{itemize}
\textbf{\underline{Conclusion:}} It has a \textbf{LOW} complexity with 0 RET/FTR and 14 DET.

\subsection{Add new categories}
\textbf{Internal Logic File (ILF)} \\ 
\textbf{RET/FLR}
\begin{itemize}
\item Application database.
\end{itemize}
\textbf{DET}
\begin{itemize}
\item Name of the new category.
\item ID of the new category.
\item ID of the person in charge of the new department.
\item Description of the new category.
\item Priority level of the new category.

\end{itemize}
% % % I thought of 8 including chief's information. If we just add with an ID thats great.
\textbf{\underline{Conclusion:}} It has a \textbf{LOW} complexity with 1 RET/FTR and 5 DET.

\subsection{Delete categories}
\textbf{Internal Logic File (ILF)} \\ 
\textbf{RET/FLR}
\begin{itemize}
\item None.
\end{itemize}
\textbf{DET}
\begin{itemize}
\item Category to delete.
\end{itemize}
\textbf{\underline{Conclusion:}} It has a \textbf{LOW} complexity with 0 RET/FTR and 1 DET.

\subsection{Modifying category stored values and its priority}
\textbf{Internal Logic File (ILF)} \\ 
\textbf{RET/FLR}
\begin{itemize}
\item Application database.
\end{itemize}
\textbf{DET}
\begin{itemize}
\item Category to modify.
\item New priority of the category to modify.
\end{itemize}
\textbf{\underline{Conclusion:}} It has a \textbf{LOW} complexity with 1 RET/FTR and 2 DET.	



\section{Report System}



\subsection{Fault location}
\textbf{External Input (EI)} \\ 
\textbf{RET/FLR}
\begin{itemize}
\item None.
\end{itemize}
\textbf{DET}
\begin{itemize}
\item Location of the fault (automatically or manually chosen).
\end{itemize}
\textbf{\underline{Conclusion:}} It has a \textbf{LOW} complexity with 0 RET/FTR and 1 DET.

\subsection{Incidences report}
\textbf{External Input (EI)} \\ 
\textbf{RET/FLR}
\begin{itemize}
\item Application database.
\end{itemize}
\textbf{DET}
\begin{itemize}
\item Location of the fault.
\item Category of the fault.
\item Subject of the fault.
\item Description of the fault.
\item Status of the fault.
\item Priority of the fault.
\item Timestamp of the report.
\item Photography of the fault.
\item ID of the reporter.
\end{itemize}
\textbf{\underline{Conclusion:}} It has a \textbf{LOW} complexity with 1 RET/FTR and 9 DET.

\subsection{Store report}
\textbf{Internal Logic File (ILF)} \\ 
\textbf{RET/FLR}
\begin{itemize}
\item Application database.
\end{itemize}
\textbf{DET}
\begin{itemize}
\item ID of the fault.
\item Location of the fault.
\item Category of the fault.
\item Subject of the fault.
\item Description of the fault.
\item Status of the fault.
\item Priority of the fault.
\item Timestamp of the report.
\item Photography of the fault.
\item ID of the reporter.
\end{itemize}
\textbf{\underline{Conclusion:}} It has a \textbf{LOW} complexity with 1 RET/FTR and 10 DET.


\section{Notification and Messaging System}

\subsection{Send general notification}
\textbf{External Query (EQ)} \\ 
\textbf{RET/FLR}
\begin{itemize}
\item Application database.
\end{itemize}
\textbf{DET}
\begin{itemize}
\item List of technicians affected.
\item List of reporters affected.
\item List of chiefs affected.
\item List of faults (assigned or deleted or completed...)
\item Name of category deleted.
\item Name of category created.
\item Identifier of the notification received.
\item Description.
\end{itemize}
\textbf{\underline{Conclusion:}} It has a \textbf{LOW} complexity with 1 RET/FTR and 8 DET.


\subsection{Receive general notification}
\textbf{External Query (EQ)} \\ 
\textbf{RET/FLR}
\begin{itemize}
\item Application database.
\end{itemize}
\textbf{DET}
\begin{itemize}
\item List of technicians affected.
\item List of reporters affected.
\item List of chiefs affected.
\item List of faults (assigned or deleted or completed...)
\item Name of category deleted.
\item Name of category created.
\item Identifier of the notification received.
\item Description.

\end{itemize}
\textbf{\underline{Conclusion:}} It has a \textbf{LOW} complexity with 1 RET/FTR and 8 DET.

\subsection{Send messages to users}
\textbf{Internal Logic File (ILF)} \\ 
\textbf{RET/FLR}
\begin{itemize}
\item Application database.
\item Message to send.
\end{itemize}
\textbf{DET}
\begin{itemize}
\item Fault's ID in which conversation takes place.
\item Reporter's ID.
\item Technician's ID.
\item Description message.
\end{itemize}
\textbf{\underline{Conclusion:}} It has a \textbf{LOW} complexity with 1 RET/FTR and 4 DET.

\subsection{Read messages}
\textbf{Internal Logic File (EQ)} \\ 
\textbf{RET/FLR}
\begin{itemize}
\item Application database.
\end{itemize}
\textbf{DET}
\begin{itemize}
\item Fault's ID in which conversation takes place.
\item Reporter's ID.
\item Technician's ID.
\item Description message.
\end{itemize}
\textbf{\underline{Conclusion:}} It has a \textbf{LOW} complexity with 1 RET/FTR and 4 DET.



\section{User Management}

\subsection{Login into the system}
\textbf{External Interface Files (EIF)} \\ 
\textbf{RET/FLR}
\begin{itemize}
\item None.
\end{itemize}
\textbf{DET}
\begin{itemize}
\item UAM's credential: email.
\item UAM's credential: password.
\end{itemize}
\textbf{\underline{Conclusion:}} It has a \textbf{LOW} complexity with 0 RET/FTR and 2 DET.

\subsection{Report a misbehaving user}
\textbf{Internal Logic File (ILF)} \\ 
\textbf{RET/FLR}
\begin{itemize}
\item Application database.

\end{itemize}
\textbf{DET}
\begin{itemize}
\item Misbehaving user's ID.
\item Description of the reason of reporting the user.
\item Reporter's ID.

\end{itemize}
\textbf{\underline{Conclusion:}} It has a \textbf{LOW} complexity with 1 RET/FTR and 11 DET.

\subsection{Ban an user from the system}
\textbf{Internal Logic File (ILF)} \\ 
\textbf{RET/FLR}
\begin{itemize}
\item Application database.
\end{itemize}
\textbf{DET}
\begin{itemize}
\item User's ID.
\end{itemize}
\textbf{\underline{Conclusion:}} It has a \textbf{LOW} complexity with 1 RET/FTR and 1 DET.

\subsection{See fault history}
\textbf{External Query (EQ)} \\ 
\textbf{RET/FLR}
\begin{itemize}
\item List of incidences in the application database, related to an user,
\end{itemize}
\textbf{DET}
\begin{itemize}
\item ID of the fault.
\item Location of the fault.
\item Category of the fault.
\item Subject of the fault.
\item Description of the fault.
\item Status of the fault.
\item Priority of the fault.
\item Timestamp of the report.
\item Photography of the fault.
\item ID of the reporter.
\end{itemize}
\textbf{\underline{Conclusion:}} It has a \textbf{LOW} complexity with 1 RET/FTR and 10 DET.

\subsection{Show the list of members of technical staff}
\textbf{External Query (EQ)} \\ 
\textbf{RET/FLR}
\begin{itemize}
\item List of technicians stored in the application database.
\end{itemize}
\textbf{DET}
\begin{itemize}
\item ID of each technician.
\item Name and surname of each technician.
\item Telephone number of each technician.
\item Position/Rank (inside the department) of each technician.
\item Number of repair tasks solved by each technician.
\item Career of each technician (years). 
\end{itemize}
\textbf{\underline{Conclusion:}} It has a \textbf{LOW} complexity with 1 RET/FTR and 6 DET.

\subsection{Show detailed information about technician}
\textbf{External Query (EQ)} \\ 
\textbf{RET/FLR}
\begin{itemize}
\item Technician information from the application database.
\end{itemize}
\textbf{DET}
\begin{itemize}
\item ID of the technician.
\item Name and surname of the technician.
\item Department and position/rank inside that department.
\item List of repair tasks solved by the technician.
\item List of repair tasks currently assigned to the technician.
\end{itemize}
\textbf{\underline{Conclusion:}} It has a \textbf{LOW} complexity with 1 RET/FTR and 5 DET.



\section{Fault History and Statistics}

\subsection{Check history of reported fault}
\textbf{External Query (EQ)} \\ 
\textbf{RET/FLR}
\begin{itemize}
\item Report fault from the application database.
\end{itemize}
\textbf{DET}
\begin{itemize}
\item ID of the fault.
\item Location of the fault.
\item Category of the fault.
\item Subject of the fault.
\item Description of the fault.
\item Current status of the fault and its timestamp of last change.
\item Priority of the fault.
\item Timestamp of the report.
\item Photography of the fault.
\item ID of the reporter.
\item ID of the assigned technician.
\end{itemize}
\textbf{\underline{Conclusion:}} It has a \textbf{LOW} complexity with 1 RET/FTR and 11 DET.

\subsection{Tracking incidence impact}
\textbf{External Query (EQ)} \\ 
\textbf{RET/FLR}
\begin{itemize}
\item Report fault from the application database.
\end{itemize}
\textbf{DET}
\begin{itemize}
\item ID of the fault.
\item Location of the fault.
\item Category of the fault.
\item Subject of the fault.
\item Description of the fault.
\item Status of the fault.
\item Priority of the fault.
\item Timestamp of the report.
\item Photography of the fault.
\item ID of the reporter.
\item ID of the assigned technician.
\item Numeric value of its impact (importance).
\end{itemize}
\textbf{\underline{Conclusion:}} It has a \textbf{LOW} complexity with 1 RET/FTR and 12 DET.

\subsection{Get report of incidences}
\textbf{External Query (EQ)} \\ 
\textbf{RET/FLR}
\begin{itemize}
\item Report of incidences from the application database.
\end{itemize}
\textbf{DET}
\begin{itemize}
\item ID of each fault.
\item Location of each fault.
\item Category of each fault.
\item Subject of each fault.
\item Description of each fault.
\item Status of each fault.
\item Priority of each fault.
\item Timestamp of each fault report.
\item Photography of each fault.
\item ID of each fault's reporter.
\item ID of each fault's technician.
\end{itemize}
\textbf{\underline{Conclusion:}} It has a \textbf{LOW} complexity with 1 RET/FTR and 11 DET.

\subsection{List all reports}
\textbf{External Query (EQ)} \\ 
\textbf{RET/FLR}
\begin{itemize}
\item List of all fault reports in the application database.
\end{itemize}
\textbf{DET}
\begin{itemize}
\item ID of each fault.
\item Location of each fault.
\item Category of each fault.
\item Status of each fault.
\item Priority of each fault.
\item Timestamp of each fault report.
\end{itemize}
\textbf{\underline{Conclusion:}} It has a \textbf{LOW} complexity with 1 RET/FTR and 6 DET.


\subsection{Visualize history of tasks}
\textbf{External Query (EQ)} \\ 
\textbf{RET/FLR}
\begin{itemize}
\item List of all repair tasks from the database.
\end{itemize}
\textbf{DET}
\begin{itemize}
\item ID of each task.
\item Location of each task.
\item Category of each task.
\item Status of each task.
\item Priority of each task.
\item Timestamp of each task.
\end{itemize}
\textbf{\underline{Conclusion:}} It has a \textbf{LOW} complexity with 1 RET/FTR and 6 DET.

\subsection{Sort history of tasks}
\textbf{External Output (EO)} \\ 
\textbf{RET/FLR}
\begin{itemize}
\item Application database.
\item List of task to be sorted.
\end{itemize}
\textbf{DET}
\begin{itemize}
\item ID of each fault.
\item Location of each fault.
\item Category of each fault.
\item Subject of each fault.
\item Description of each fault.
\item Status of each fault.
\item Priority of each fault.
\item Timestamp of each fault report.
\item Photography of each fault.
\item ID of each fault's reporter.
\item ID of each fault's technician.
\end{itemize}
\textbf{\underline{Conclusion:}} It has a \textbf{MEDIUM} complexity with 2 RET/FTR and 11 DET.

\subsection{Consult extend information about an incidence on the report}
\textbf{External Query (EQ)} \\ 
\textbf{RET/FLR}
\begin{itemize}
\item Detailed information about an specific indicence from the report.
\end{itemize}
\textbf{DET}
\begin{itemize}
\item ID of the fault.
\item Location of the fault.
\item Category of the fault.
\item Subject of the fault.
\item Description of the fault.
\item Status of the fault.
\item Priority of the fault.
\item Timestamp of the report.
\item Photography of the fault.
\item ID of the reporter.
\item ID of the assigned technician.
\end{itemize}
\textbf{\underline{Conclusion:}} It has a \textbf{LOW} complexity with 1 RET/FTR and 11 DET.

\subsection{Generate statistics about incidences}
\textbf{External Output (EO)} \\ 
\textbf{RET/FLR}
\begin{itemize}
\item Application database.
\end{itemize}
\textbf{DET}
\begin{itemize}
\item Technicians ranking, ordered by department and number of solved tasks.
\item Department ranking, ordered by total number of incidences.
\item UAM buildings ranking, ordered by total number of incidences happened in them.
\item User reporting ranking, ordered by total number of incidences reported.
\item Line/Bar graphs showing number of incidences, organized by departments and months. There are two graphs for each of the 8 departments: one showing the last 12 months and the other is global (since the system was implanted). (2 x 8 = 16 graphs)
\end{itemize}
\textbf{\underline{Conclusion:}} It has a \textbf{MEDIUM} complexity with 1 RET/FTR and 20 DET.

\subsection{Generate PDF file from a report}
\textbf{External Output (EO)} \\ 
\textbf{RET/FLR}
\begin{itemize}
\item Report generated using information from the application database.
\end{itemize}
\textbf{DET}
\begin{itemize}
\item Technicians ranking, ordered by department and number of solved tasks.
\item Department ranking, ordered by total number of incidences.
\item UAM buildings ranking, ordered by total number of incidences happened in them.
\item User reporting ranking, ordered by total number of incidences reported.
\item Line/Bar graphs showing number of incidences, organized by departments and months. There are two graphs for each of the 8 departments: one showing the last 12 months and the other is global (since the system was implanted). (2 x 8 = 16 graphs)
\end{itemize}
\textbf{\underline{Conclusion:}} It has a \textbf{MEDIUM} complexity with 1 RET/FTR and 20 DET.


\chapter{Cocomo II Cost Estimation}
\label{chapCocomo}

Table \ref{tblCocomoDetail} reflects the estimation of costs for the smallest subsystem, the fault reporting. The size parameters has been extracted from table \ref{tbl_RSS_UFP}.

\begin{table}[hbtp]
\centering
\begin{tabular}{l|c|cccc}
\multirow{2}{*}{\textbf{Phase}} & \textbf{Effort} & \textbf{Labour} & \textbf{Running} & \textbf{Fixed} & \textbf{Total} \\
& \textit{Person-months} & \textit{K\euro} & \textit{K\euro / month} & \textit{K\euro} & \textit{K\euro} \\
\textsc{Requirements} & 0.7 & 0.8 & 1.05 & 0 & 0.905  \\
\textsc{Product design} & 0.3 & 1.5 & 1.05 & 0 & 1.815  \\
\textsc{Detailed design} & 0.4 & 2.3 & 1.05 & 0 & 2.72  \\
\textsc{Code \& Unit tests} & 0.5 & 2.8 & 1.05 & 8.25 & 11.575  \\
\textsc{Integration \& test} & 0.3 & 1.6 & 1.05 & 3.2 & 5.115  \\ \hline
\textsc{Development} & 1.5 & 8.2 & 1.05 & 11.45 & 21.225  \\
\textsc{Totals} & 1.6 & 9 & 1.05 & 11.45 & 22.13
\end{tabular}

\caption{Detailed report of the estimation using the CoCoMo II method}
\label{tblCocomoDetail}
\end{table}

The estimation approximately coincides with the one given based on the function points method ($\sim$ 9,000 \euro\ in labour costs). It's worth noting that this estimation has been calculated as if it were the first prototype of the project - it includes software and hardware costs, which are unique for the project - and that it doesn't take into account work parallelization.

\section{COCOMO II parameters}


% % I have used http://csse.usc.edu/csse/research/COCOMOII/cocomo2000.0/CII_modelman2000.0.pdf to invent justifications of parameters.


\subsection{Personnel parameters}
\paragraph{Analyst Capability (ACAP) } \textit{High}

The analyst assigned to the project are the best ones we have. They work almost without mistakes, they are above 80 percentil, according to CMMI.

\paragraph{Programmer Capability (PCAP) } \textit{High}

Our programmers are very well formed and they work with very few mistakes, they are above 80 percentil, according to CMMI.

\paragraph{Personnel Continuity (PCON) } \textit{Nominal}

We were in 14\% of project's annual turnovers during 2013 and 13.35\% during 2014.

\paragraph{Applications Experience (APEX) } \textit{Nominal}

As this is not a very complex system, we want some new people to work with experienced ones, so they can learn better how we work in Triforce. When we are talking about new people, we talk about people with experience between 1 year and 2 that had been working in some similar project. 


\paragraph{Platform Experience (PLEX) } \textit{High}

We are very used to the platform we will work with. 

\paragraph{Language and Tool Experience (LTEX) } \textit{Nominal}

We are not very used to new standard HTML5, but we are with other languages, such as SQL and JavaScript.


\subsection{Project parameters}
\paragraph{Use of Software Tools (TOOL) } \textit{Very High}

We have been working in Software Engineering for the past 15 years so we are very very used with Software tools.

\paragraph{Multisite Development (SITE) } \textit{High}

The team will work all of them in the same city, but not all of them in the same building.

\paragraph{Required Development Schedule (SCED) } \textit{Nominal} - The plan will be followed with no delays or overtakes. 

\subsection{Product parameters}

\paragraph{Required Software Reliability (RELY) }\textit{ Low}

The possible losses are easily recoverable.


\paragraph{Data Base Size (DATA) }\textit{ Nominal}

D/P coefficient using approximate database size estimation is 43 (between 10 and 100)


\paragraph{Product Complexity (CPLX) }\textit{ Low }

The program will have straightforward nesting of structured programming operators.  No cognizance of particular processor or I/O device characteristics is needed. We will use of simple graphic user interface (GUI) builders. 


\paragraph{Developed for Reusability (RUSE):}  \textit{Nominal}

We won't focus on developing for re-usability but in requirements, but we won't make effort in developing uniquely so it won't be reusable.

\paragraph{Documentation Match to Life-Cycle Needs (DOCU)} \textit{nominal} 

Documentation is important but we have been debugging how much effort we spend in documentation and we discovered the optimum amount of documentation we need for maintenance.

\subsection{Platform parameters}
\paragraph{Execution Time Constraint  (TIME)} \textit{nominal} 

We won't focus on execution time. It's not a very complex system to use loads of resources but it has many functionalities.

\paragraph{Main Storage Constraint (STOR)} \textit{nominal}

We won't focus on Main Storage and we estimate the system won't need more than 50\% of the Main Memory.


\paragraph{Platform Volatility (PVOL) } \textit{low}

Just OS updates or new software releases for database. This updates will be taken into account less frequently than once a year.



\chapter{Gantt Diagram}
\label{chapGantt}
Included below is the Gantt diagram generated

\includepdf[scale=0.8, pages={1-}, nup=1x2, pagecommand={}]{../Project.pdf}

\end{document}
