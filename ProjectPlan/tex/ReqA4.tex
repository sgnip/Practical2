
The system provides a priorities categorization manager which shows the whole list of categories with their assigned priority.

\begin{requirement}{Filtering list of categories}
\reqdesc The system must allow users to search and filter from the category list.
\reqin Optional external input:
\begin{itemize}
	\item Value to be searched in the whole list.
\end{itemize}
\reqout \begin{itemize}
	\item The list of categories will be shown.
	\item External Query: the user will be able to filter the list giving values interactively.
	\item Category object \textbf{if one of the categories} from the list \textbf{is selected}.
\end{itemize}
\end{requirement}


\begin{requirement}{Add new categories}
\reqdesc  The system must allow administrators to add new categories.
\reqin \begin{itemize}
	\item Information needed to create category. This information is given by external inputs:
		\subitem Title.
		\subitem Information of department's chief. 
\end{itemize}
\reqout \begin{itemize}
	\item OK or Error message.
	\item if OK:
		\subitem Department chief object stored into database.
		\subitem List of categories updated.
\end{itemize}
\reqsteps Assure the category does not exist previously before creating it.
\end{requirement}

\begin{requirement}{Delete categories}
\reqdesc  The system must allow administrators to delete categories.
\reqin Category object.
\reqout \begin{itemize}
	\item OK or Error message.
	\item if OK:
		\subitem List of categories updated.
\end{itemize}
\end{requirement}


\begin{requirement}{Modifying stored values}
\reqdesc The system must allow administrators to modify stored values in all attributes
\reqin 
\begin{itemize}
\item One of the following:
\subitem Reported fault object
\subitem User object (including all sub-kinds of user's object such as Department chief's object)
\item The attribute of the selected object to be modified.
\item The new value to be stored.
\end{itemize}

\reqout \begin{itemize}
	\item OK or Error message.
	\item External query to select the attribute, because the attribute depends on the object the user wants to modify.
\end{itemize}
\end{requirement}


\begin{requirement}{List all reports}\label{A4-ListAllReports}
\reqdesc The system must allow listing all reported faults.
\reqin 
\begin{itemize}
 	\item Some (or none) values to filter the list by, such as:
 		\subitem Category.
 		\subitem Priority.
 		\subitem Location.
 \end{itemize}
\reqout External \textbf{query}: List of the reports matching with given values, ordered mainly by category and secondly by date. This list can be updated interactively by the user modifying filter values.
\reqsteps The system will search into the database all reports matching the given values. If a value is changed, the list must be filtered again with the new given values.
\end{requirement}

\begin{requirement}{Consult single report}
\reqdesc Users will be able to check details of a reported fault from the full list of reports (given by \ref{A4-ListAllReports}).
\reqin Fault reported object with all information attached.
\reqout Report showing all the information attached to the object.
\end{requirement}

\begin{requirement}{Modify single report's priority}
\reqdesc The system must allow modifying the priority given to a single report. 
\reqin \begin{itemize}
\item The report to be updated.
\item The new priority to be stored.
\end{itemize}
\reqout \begin{itemize}
	\item OK or Error message.
	\item New reported fault object with new information modified.
	\item Updating database with new information.
	\item Notification to all users related to the task (reporter and maintenance personnel assigned to it)
\end{itemize}

\end{requirement}




