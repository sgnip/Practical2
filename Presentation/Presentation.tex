\documentclass[10pt, compress,usetitleprogressbar,aspectratio=1610]{beamer}

\usetheme{m}
\usepackage[scale=2]{ccicons}
\usepackage{minted}
\usepackage{hyperref}

\usemintedstyle{trac}

% Presentation theme: https://github.com/matze/mtheme
% Requirements:
% - Xelatex
% - Fira fonts (http://www.carrois.com/fira-4-0/#download, install Sans and Mono)
% - Pygments (sudo pip install pygments, if pip is not installed do sudo apt-get install python-pip)
% - The Makefile takes care of the compilation options. Make continuous adds the -pvc option for continuous preview

\hypersetup{
  hyperindex,
    colorlinks,
    allcolors=blue!60!black
}

\title{Fault Manager Lite}
\subtitle{Project plan - Goals, estimations and planning}
\date{\today}
\author{Iván Márquez Pardo \and Víctor de Juan Sanz \and Guillermo Julián Moreno}
\institute{Triforce}

\begin{document}

\maketitle

\tableofcontents

\section{Introduction}
\begin{frame}[fragile]
\frametitle{Project definition}

Fault Manager Lite is the future system that will fulfill the necessities of the UAM regarding notification and management of faults that arise on campus.

FML unifies a fault report system with a task manager with the aim of speeding up fault troubleshooting. Users will be able to report faults in \emph{less than two minutes}. Technicians will automatically be assigned on them based on their location and department.
\end{frame}

\section{Project definition}

\begin{frame}
\frametitle{Goals}

\begin{itemize}
\item \textbf{Reduce fault detection time} in order to facilitate a quick troubleshooting.
\item \textbf{Efficient allocation of resources} relying on automatic systems and crowdsourced reports instead of manual management.
\item \textbf{Coordination} of all the maintenance staff.
\item \textbf{Facilitate reports} in a way that users can communicate what's wrong without spending too much time.
\end{itemize}
\end{frame}

\begin{frame}
\frametitle{Subsystems (I)}
\begin{itemize}
\item \alert{Task management} Responsible for receiving and analyzing reports. Aspects like task assignments, priority categorization and repair status will be managed by this subsystem.
\item \alert{Reporting} Will allow users to send reports about discovered faults. It will also detect duplicates before sending them to the task manager.
\item \alert{Notification \& messaging} Responsible for notifying technicians and users about emergencies, changes and updates in reports. It will also allow private communication channels between reporters and technicians.
\end{itemize}
\end{frame}

\begin{frame}
\frametitle{Subsystems (II)}
\begin{itemize}
\item \alert{User management} Communicates with the UAM login server in order to authenticate users. Keeps a profile on each user who registered within the system, their role and their fault report history.
\item \alert{Fault history \& stats} Tracks changes in the fault reports and generates statistics based on them, either for the whole system, for each department or even for a specific user of the system. It will also allow visualization of all the reported faults in a campus map.
\end{itemize}
\end{frame}


\end{document}
